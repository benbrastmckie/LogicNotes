%!TEX root = ../LogicNotes.tex
%----------------------------------------------------------------------------------------
%   ADDITIONAL MATERIALS
%   Created: 2025-12-02
%   Description: Supplementary sections on alternative semantic approaches
%----------------------------------------------------------------------------------------
%
% This file contains sections that were moved from the main LogicNotes.tex file
% to serve as supplementary material. These sections present alternative semantic
% approaches that complement the main pedagogical narrative.
%
%----------------------------------------------------------------------------------------

\section*{\sc Propositional Tense Logic: Indeterminacy}
  \label{sec:Indeterminacy}

\begin{enumerate}[leftmargin=1.2in]
	\item[\bf Well-Formed Sentences:] Letting $p_i \in \SL$, the \textit{well-formed sentences} $\TL_\square$ are defined:
    \[ \metaA \Coloneq p_i \mid
      \bot \mid
      \neg\metaA \mid
      \metaA \rightarrow \metaA \mid
      \Past\metaA \mid
      \Future\metaA \mid
      \Inevitably\metaA
    \]
    Define $\wfs{\TL_\square}$ to be the set of all well-formed sentences of $\TL_\square$.
  \item[\bf Abbreviation:] In addition to the abbreviations for $\TL$, we may define $\inevitably\metaA \coloneq \neg\Inevitably\neg \metaA$.
  \item[\bf Index:] Let $\T_i = \tuple{T_i, <_i}$ share the same index.
  \item[\bf Subframe:] $\T_i$ is a \textit{subframe} of $\T_j$ \textit{iff} $T_i \subseteq T_j$ and $x <_j y$ whenever $x <_i y$.
  \item[\bf History:] $\T_i$ is a \textit{history} of the frame $\T$ \textit{iff:} (1) $\T_i$ is a subframe of $\T$ that satisfies \textsc{tot}; and (2) for every subframe $\T_j$ of $\T$ that satisfies \textsc{tot}, if $\T_i$ is a subframe of $\T_j$, then $\T_i = \T_j$ (i.e., $\T_i$ is a \textit{maximal total subframe} of $\T$).
  \item[\bf Possible Histories:] Let $H_\T$ be the set of all histories of $\T$.
  % \item[\bf Inclusion:] We may take $x \in \T_i \coloneq (x \in T_i) \wedge (\T_i = \tuple{T_i, <_i})$ for convenience.
  \item[\bf Inevitability Set:] Let $H_\T^x \coloneq \set{\T_i \in H_\T \mid x \in T_i}$ be the histories of $\T$ in which $x$ occurs.
  \item[\bf Language $\boldsymbol{\TL_\square}$:] Let $\TL_\square$ extend $\TL$ to include `$\Inevitably$' where $\corner{\Inevitably\metaA}$ reads $\ulcorner$Inevitably $\metaA\urcorner$.
  \item[\bf Inevitability:] We define $\vDash$ recursively for a model $\M = \tuple{T, <, \I}$ of $\TL$, history $\T_i = \tuple{T_i, <_i}$, times $x, y \in T$, sentence letter $p_i \in \SL$, and $\metaA, \metaB \in \wfs{\TL}$:
    \begin{itemize}[leftmargin=.15in]\small
      \item[] $\M, \T_i, x \nvDash \bot$.
      \item[] $\M, \T_i, x \vDash p_i$ \textit{iff} $x \in \I(p_i)$.
      \item[] $\M, \T_i, x \vDash \neg \metaA$ \textit{iff} $\M, \T_i, x\nvDash \metaA$.
      \item[] $\M, \T_i, x \vDash \metaA\rightarrow \metaB$ \textit{iff} $\M, \T_i, x\nvDash \metaA$ or $\M, \T_i, x \vDash  \metaB$.
      \item[] $\M, \T_i, x \vDash \Past \metaA$ \textit{iff} $\M, \T_i, y\vDash \metaA$ for every $y \in T_i$ such that $y <_i x$.
      \item[] $\M, \T_i, x \vDash \Future \metaA$ \textit{iff} $\M, \T_i, y\vDash \metaA$ for every $y \in T_i$ such that $x <_i y$.
      \item[] $\M, \T_i, x \vDash \Inevitably \metaA$ \textit{iff} $\M, \T_j, x\vDash \metaA$ for every $\T_j \in H_\T^x$.
    \end{itemize}
  \item[\bf Logical Consequence:] $\MetaG \MLmodels[] \metaA$ just in case for any model $\M = \tuple{T, <, \I}$ of $\TL_\square$, history $\T_i \in H_\T$, and time $x \in T$, if $\M, \T_i, x \vDash \metaG$ for all $\metaG \in \MetaG$, then $\M, \T_i, x \vDash \metaA$.
  % \item[\bf Truth-Value Gaps:] Thomason defines \textit{truth} relative to a model and time as:
  %   \begin{itemize}[leftmargin=.15in]\small
  %     \item[] $\M, x \vDash \metaA$ \textit{iff} $\M, \T_i, x \vDash \metaA$ for all $\T_i \in H_\T^x$.
  %     \item[] $\M, x \Dashv \metaA$ \textit{iff} $\M, \T_i, x \vDash \metaA$ for all $\T_i \in H_\T^x$.
  %   \end{itemize}
  %   % Otherwise neither, e.g, $\Future\metaA \vee \Future\neg\metaA$.
  % \item[\bf Logical Consequence:] $\MetaG \MLmodels[] \metaA$ just in case for any model $\M = \tuple{T, <, \I}$ of $\TL_\square$ and time $x \in T$, if $\M, x \vDash \metaG$ for all $\metaG \in \MetaG$, then $\M, x \vDash \metaA$.
\end{enumerate}




\section*{\it Problem Set: Indeterminacy}

\begin{enumerate}[leftmargin=1.2in]
  \item[\bf Logical Consequence:] Without imposing any restriction on the models of $\TL_\square$, evaluate the following where $p_i \in \SL$, providing a proof or countermodel:
    \begin{enumerate}[label=\arabic*.,resume]\small
      \begin{multicols}{2}
        \item $\MLmodels[] p_i \rightarrow \inevitably p_i$.
        \item $\MLmodels[] \metaA \rightarrow \inevitably\metaA$.
        \item $\MLmodels[] \past\metaA \vee \past\neg\metaA$.
        \item $\MLmodels[] \metaA \rightarrow \Future\past\metaA$.
        \item $\MLmodels[] \Past\Future\metaA \rightarrow \always\metaA$.

        \item $\MLmodels[] p_i \rightarrow \Inevitably p_i$.
        \item $\MLmodels[] \metaA \rightarrow \Inevitably\metaA$.
        \item $\MLmodels[] \future\metaA \vee \future\neg\metaA$.
        \item $\MLmodels[] \metaA \rightarrow \Past \future \metaA$.
        \item $\MLmodels[] \Future\Past\metaA \rightarrow \always\metaA$.
      \end{multicols}
    \end{enumerate}
\end{enumerate}



\section*{\sc Propositional Bimodal Logic: Cartesian Semantics}
  \label{sec:Cartesian}

\begin{enumerate}[leftmargin=1.2in]
	\item[\bf Well-Formed Sentences:] Letting $p_i \in \SL$, the \textit{well-formed sentences} $\BLC$ are defined:
    \[ \metaA \Coloneq p_i \mid
      \bot \mid
      \neg\metaA \mid
      \metaA \rightarrow \metaA \mid
      \Past\metaA \mid
      \Future\metaA \mid
      \boxtimes\metaA \mid
      \Box\metaA
    \]
    Define $\wfs{\BLC}$ to be the set of all well-formed sentences of $\BLC$.
  \item[\bf Abbreviation:] We maintain the abbreviations from $\PL$, $\ML$, and $\TL$ along with the convention
    $\Q\Gamma \coloneq \set{\Q\gamma : \gamma \in \Gamma}$ for any $\Q \in \set{\Past, \Future, \boxtimes, \Box}$.
    % where $\metaA_{\tuple{\textsc{p} | \textsc{f}}}$ is the result of exchanging all occurrences of $\Past$ and $\Future$ in $\metaA$, the \textit{Logic of Tense and Modality} \textbf{TM} is the smallest extension of the set of classical propositional tautologies \textbf{PL} to be closed under all instances of the schemata following:
  \item[\bf Frame:] A \textit{Cartesian frame} is an ordered quadruple $\C = \tuple{W, R, T, <}$ where $\tuple{W, R}$ is a modal frame and $\tuple{T, <}$ is a temporal frame.
  \item[\bf Interpretation:] A \textit{Cartesian interpretation} of $\BLC$ over $\C$ is a function $\I : \SL \to \wp(W \times T)$, i.e., where $\I(p_i) \subseteq W \times T$ for every sentence letter $p_i \in \SL$.
  \item[\bf Model:] A Cartesian model of $\BLC$ is an $\M = \tuple{W, R, T, <, \I}$ where $\tuple{W, R, T, <}$ is a Cartesian frame and $\I$ is a Cartesian interpretation of $\BLC$.
  \item[\bf Semantics:] We define $\vDash$ recursively for model $\M = \tuple{W, R, T, <, \I}$ of $\BLC$, worlds $w, u \in W$, times $x, y \in T$, sentence letter $p_i \in \SL$, and $\metaA, \metaB \in \wfs{\BLC}$:
    \begin{itemize}[leftmargin=.15in]\small
      \item[] $\M, w, x \nvDash \bot$.
      \item[] $\M, w, x \vDash p_i$ \textit{iff} $x \in \I(p_i)$.
      \item[] $\M, w, x \vDash \neg \metaA$ \textit{iff} $\M, w,x\nvDash \metaA$.
      \item[] $\M, w, x \vDash \metaA\rightarrow \metaB$ \textit{iff} $\M, w, x\nvDash \metaA$ or $\M, w, x \vDash  \metaB$.
      \item[] $\M, w, x \vDash \Past \metaA$ \textit{iff} $\M, w, y\vDash \metaA$ for every $y \in T$ such that $y < x$.
      \item[] $\M, w, x \vDash \Future \metaA$ \textit{iff} $\M, w, y\vDash \metaA$ for every $y \in T$ such that $x < y$.
      \item[] $\M, w, x \vDash \boxtimes \metaA$ \textit{iff} $\M, u, x\vDash \metaA$ for every $u \in W$ and every $x \in T$. % where $R(w, u)$.
      \item[] $\M, w, x \vDash \Box \metaA$ \textit{iff} $\M, u, x\vDash \metaA$ for every $u \in W$. % where $R(w, u)$.
    \end{itemize}
  \item[\bf Logical Consequence:] $\MetaG \MLmodels[] \metaA$ just in case for any model $\M = \tuple{W, R, T, <, \I}$ of $\BLC$, world $w \in W$, and time $x \in T$, if $\M, w, x \vDash \metaG$ for all $\metaG \in \MetaG$, then $\M, w, x \vDash \metaA$.
\end{enumerate}



\section*{\it Problem Set: Cartesian Semantics}

\begin{enumerate}[leftmargin=1.2in]
  \item[\bf Countermodels:] Evaluate the following, providing a proof or countermodel:
    \begin{enumerate}[label=\arabic*.,resume]\small
      \begin{multicols}{2}
        \item \textit{If} $\Gamma \MLmodels[] \metaA$, \textit{then} $\Box\Gamma \MLmodels[] \Box\metaA$.
        \item $\MLmodels[] \Box\metaA \rightarrow \metaA$.
        \item $\MLmodels[] \Box\metaA \rightarrow \Box\Box\metaA$.
        \item $\MLmodels[] \metaA \rightarrow \Box\Diamond\metaA$.
        % \item $\MLmodels[] \Box\metaA \rightarrow \Box\Future\metaA$.
        \item $\MLmodels[] \Box\metaA \rightarrow \always\metaA$.

        \item $\MLmodels[] \boxtimes\metaA \leftrightarrow \Box\always\metaA$.
        \item $\MLmodels[] \boxtimes\metaA \rightarrow \metaA$.
        \item $\MLmodels[] \boxtimes\metaA \rightarrow \boxtimes\boxtimes\metaA$.
        \item $\MLmodels[] \metaA \rightarrow \boxtimes\diamondtimes\metaA$.
        % \item $\MLmodels[] \Box\metaA \rightarrow \Future\Box\metaA$.
        \item $\MLmodels[] \boxtimes\metaA \rightarrow \always\metaA$.
      \end{multicols}
    \end{enumerate}
  % \item[\bf Logical Consequence:] For each of the claims above, strengthen $\MLmodels[]$ by imposing the weakest set of constraints $C$ which make that claim valid.
\end{enumerate}



% sec:FirstOrder relocated from LogicNotes.tex on 2025-12-02
% Includes complete syntax, proof theory, semantics, and metalogic subsections

\section*{\sc First-Order Logic: Syntax}
  \label{sec:FirstOrder}

\begin{enumerate}[leftmargin=1.2in] %,label=(\arabic*)]%,label=\roman*]
	\item[\bf Language $\boldsymbol{\FI}$:] The first-order language $\FI$ includes: constants `$c_1$', `$c_2$', \dots, variables `$x_1$', `$x_2$', \dots, $n$-place predicates `$p_1^n$', `$p_2^n$', \dots, for each natural number $n\geq0$, sentential operators `$\neg$',`$\rightarrow$', `$\forall$', and parentheses `$($' and `$)$'.
	\item[\bf Terms:] A symbol is a \textit{term} just in case that symbol is a constant or variable.
	\item[\bf Well Formed Formulas:] Let `$t_1$',\dots, `$t_n$' be terms of $\FI$, `$x$' be a variable of $\FI$, `$H^n$' be an $n$-place predicate of $\FI$, and `$A$' and `$B$' name arbitrary sentences of $\FI$. We may then let $\mathcal{G}_1$ be the set of wff of $\FI$, defined recursively as follows:
	      \begin{itemize}
		      \item The 0-place predicates `$p_1^0$',`$p_2^0$',\dots are all wff of $\FI$.
		      \item If ${H^n}$ is an $n$-place predicate of $\FI$, and ${t_1},\dots,{t_n}$ are terms of $\FI$, then the atomic sentence $\corner{H^n(t_1,\dots,t_n)}$ is a wff of $\FI$.
		      \item If ${A}$ is a wff of $\FI$, then $\corner{\neg A}$ is a wff of $\FI$.
		      \item If ${A}$ and ${B}$ are wffs of $\FI$, then $\corner{(A\vee B)}$ is a wff of $\FI$.
		      \item If ${A}$ is a wff of $\FI$, then $\corner{\forall xA}$ is a wff of $\FI$.
	      \end{itemize}
	\item[\bf Abbreviations:] (i) $\corner{(A\wedge B)}$ abbreviates $\corner{\neg(\neg A\vee\neg B)}$;\\ (ii) $\corner{(A\rightarrow B)}$ abbreviates $\corner{(\neg A\vee B)}$;\\ (iii) $\corner{(A\leftrightarrow B)}$ abbreviates $\corner{[(A\rightarrow B)\wedge(B\rightarrow A)]}$;\\ (iv) $\corner{\exists xA}$ abbreviates $\corner{\neg\forall x\neg A}$.
\end{enumerate}



\section*{\it Problem Set: Metalinguistic Abbreviation}

Let $\FI$ include the symbols in $\FI$ together with the sentential operators `$\wedge$', `$\rightarrow$', `$\leftrightarrow$', and `$\exists x_i$' which are to be read `and', `(materially) implies that', `just in case', and `every $x_i$ is such that', respectively. Provide a definition $\mathcal{G}_1^+$ of the wfss of $\FI$.




\section*{\sc First-Order Logic: Proof Theory}

%define atomic, and use this to define the free variables as those that occur in atomic wfs.
%"All occurrences of free variables y in φ are also free in ∀xφ i y is distinct from x. All other occurrences of variables are not free." Volker p.60 from Syntax and Circularity: A Study in Self-Reference and Paradox (with Graham Leigh)
\begin{enumerate}[leftmargin=1.2in] %,label=(\arabic*)]%,label=\roman*]
	\item[\bf Free Variable:] Every variable which occurs in an atomic sentence of $\FI$ is \textit{free}. If $x$ is free in the wff $A$, then $x$ is \textit{bound} in the wff $\exists xA$. The wfss of $\FI$ are those wff of $\FI$ with no free variables.
	      %\item[\bf Free Variable:] A variable $x_i$ that occurs in a wfs of $\FI$ is \textit{free} just in case there is no outside quantifier binding that variable, as in: `$P^1_1(x)$', `$p_1^2(x,c)$', etc.
	\item[\bf Substitution:] For any wfs ${A}$ and terms $t$ and $k$, let $\corner{A(t/k)}$ be the wfs which result from replacing every occurrence of $k$ in the wfs ${A}$ with $t$.
	\item[\bf Available:] A term $t$ is \textit{available} (written $t^\star$) for substitution in ${A}$ iff $t$ does not occur in ${A}$ or in any premise or undischarged assumption used to prove $A$.
	      %\item[\bf Open:] A term `$t^\#$' is \textit{open} at a point line of a proof just in case `$t^\#$' does not occur anywhere previously in that proof.
	\item[\bf Rules of Inference:] Let $\mathcal{R}_1^+$ extend $\mathcal{R}^+$ to also include the following rules of inference:%, where stared terms are available, and hashed terms do not occur previously:
\end{enumerate}


% \begin{multicols}{2}\it
% Universal Introduction:\vspace{-.05in}
% \begin{equation*}
% \fitcharg{
% \formula{A(t^\star/x)}
% }{
% \formula{\forall xA(x) \hspace*{.458in}(\forall I)}
% }
% \end{equation*}
%
% Universal Elimination:\vspace{-.05in}
% \begin{equation*}
% \fitcharg{
% \formula{\forall xA(x)}
% }{
% \formula{A(t/x) \hspace*{.568in}(\forall E)}
% }
% \end{equation*}
% \end{multicols}
% \vspace{-.05in}
%
%
% \begin{multicols}{2}\it
% Existential Introduction:\vspace{-.05in}
% \begin{equation*}
% \fitcharg{
% \formula{A(t/x)}
% }{
% \formula{\exists xA(x) \hspace*{.477in}(\exists I)}
% }
% \end{equation*}
%
% Existential Elimination:\vspace{-.05in}
% \begin{equation*}
% \fitcharg{
% \formula{\exists xA(x)}
% }{
% \formula{A(t^\star/x) \hspace*{.515in}(\exists E)}
% }
% \end{equation*}
% \end{multicols}
% \vspace{-.05in}





\section*{\sc First-Order Logic: Semantics}

\begin{enumerate}[leftmargin=1.2in] %,label=(\arabic*)]%,label=\roman*]
	\item[\bf Domain:] Let the \textit{domain} $\mathcal{D}$ be a set of objects.% and $\mathcal{D}^n=\set{\tuple{d_1,\dots,d_n}:d_i\in\mathcal{D}}$.
	\item[\bf Cartesian Domain:] Let $\mathcal{D}^n$ be the set of all ordered tuples $\tuple{d_1,\dots,d_n}$ where each $d_i$ is an object in the domain $\mathcal{D}$, i.e., $\mathcal{D}^n=\set{\tuple{d_1,\dots,d_n}:d_i\in\mathcal{D}~\text{for}~1\leq i\leq n}$.
	\item[\bf Interpretation:] Let $\FImodels$ be an \textit{interpretation} of $\FI$ over $\mathcal{D}$ just in case: (i) $\FImodels(p_i^n)\subseteq\mathcal{D}^n$ for every $i\geq1$ and $n\geq0$; and (ii) $\FImodels(c_i)\in\mathcal{D}$ for every $i\geq1$.
	      %\item[\bf Extension:] The \textit{extension} of an $n$-place predicate `$p_i^n$' of $\FI$ on an interpretation $\FImodels$ over a domain $\mathcal{D}$ is the set of ordered tuples $\FImodels(p_i^n)\subseteq\mathcal{D}^n$.
	\item[\bf Assignment:] An \textit{assignment} $\underline{a}$ is a function from the variables in $\FI$ to the members of $\mathcal{D}$ such that $\underline{a}(x_i)$ is a member of the domain $\mathcal{D}$ for every $i\geq1$.
	\item[\bf Denotation:] Let $I(t)=
		      \begin{cases}
			      \FImodels(t)     & \text{if}~~ t=c_i ~~\text{for any}~~ i\geq1 \\
			      \underline{a}(t) & \text{if}~~ t=x_i ~~\text{for any}~~ i\geq1
		      \end{cases}$
	\item[\bf Variant:] The function $\underline{a}[d/x]$ is an \textit{$x$-variant} of the assignment $\underline{a}$ just in case $\underline{a}[d/x]$ differs from $\underline{a}$ at most by setting $\underline{a}[d/x](x)=d$.
	\item[\bf Model:] A \textit{model} of $\FI$ is any ordered pair $\M=\tuple{\mathcal{D},\FImodels}$, where $\mathcal{D}$ is a domain of individuals, and $\FImodels$ an interpretation over $\mathcal{D}$.
	\item[\bf Semantics:] Given a model $\M$ of $\FI$, and assignment $\underline{a}$, we may recursively define $\M,\underline{a}\vDash A$ for all wfss $A$ of $\FI$ as follows:
	      \begin{small}
		      \begin{itemize}[leftmargin=.36in]
			      \item[$(p_i)$] $\M,\underline{a}\vDash  p_i^n(t_1,\dots,t_n)$ \textit{iff} $\tuple{I(t_1),\dots,I(t_n)}\in \FImodels(p_i^n)$.
			      \item[$(\hspace{1.5pt}\exists\hspace{1.5pt})$] $\M,\underline{a}\vDash  \exists x_iA$ \textit{iff} $\M,\underline{a}[d/x_i]\vDash A$, for some $d\in\mathcal{D}$.
			      \item[$(\neg)$] $\M,\underline{a}\vDash  \neg A$ \textit{iff} $\M,\underline{a}\nvDash A$.
			      \item[$(\vee)$] $\M,\underline{a}\vDash  A\vee B$ \textit{iff} $\M,\underline{a}\vDash  A$ or $\M,\underline{a}\vDash  B$.
			            %\item[$(\wedge)$] $\PLmodel\vDash A\wedge B$ \textit{iff} $\PLmodel\vDash A$ and $\PLmodel\vDash B$.
			            %\item[$(\shortrightarrow)$] $\PLmodel\vDash A\rightarrow B$ \textit{iff} $\nvDash_{\PLmodel} A$ or $\PLmodel\vDash B$.
			            %\item[$(\leftrightarrow)$] $\PLmodel\vDash A\leftrightarrow B$ \textit{iff} $\PLmodel\vDash A$ and $\PLmodel\vDash B$, or $\nvDash_{\PLmodel} A$ and $\nvDash_{\PLmodel} B$.
		      \end{itemize}
	      \end{small}
	      It is important that in the case where $n=0$, we adopt the convention that $\FImodels(p_i^0)=\set{\varnothing}$ indicates truth, and $\FImodels(p_i^0)=\varnothing$ indicates falsity.
\end{enumerate}



\section*{\sc First-Order Logic: Metalogic}%CONTINUE

\begin{enumerate}[leftmargin=1.2in] %,label=(\arabic*)]%,label=\roman*]
	\item[\bf Truth on a Model:] $\M\vDash_1 A$ \textit{iff} $\M,\underline{a}\vDash A$ for all variable assignments $\underline{a}$.
	\item[\bf Logical Consequence:] $\MetaG\vDash_1 A$ \textit{iff} for all models $\M$, if $\M\vDash G$ for all $G\in\MetaG$, then $\M\vDash A$.
	\item[\bf Logical Equivalence:] $A\equiv_1 B$ \textit{iff} $A\vDash_1 B$ and $B\vDash_1 A$.
	\item[\bf Logical Truth:] A wfs $A$ of $\FI$ is \textit{valid} (or a logical truth) just in case $\vDash_1 A$.
	\item[\bf First-Order Logic:] The first-order formal system of natural deduction $\mathcal{F}_1^+=\tuple{\FI,\mathcal{G}_1^+,\mathcal{A}_1^+,\mathcal{R}_1^+}$ is sound and complete, where $\mathcal{A}_1^+=\varnothing$.
\end{enumerate}





\section*{\it Problem Set: First-Order Logic\footnote{I have adapted some of the following problems from \citet{Carr2013}. See also \citet{Halbach2010}.}}

\begin{enumerate}[leftmargin=1.2in]
	\item[\bf Semantics:] Provide a semantics for the wfss of $\FI$.
	\item[\bf Regimentation:] Regiment the following arguments into $\FI$.
	      \begin{enumerate}[label=(\arabic*)]\small
		      %\item $\forall xPx\vdash \forall yPy$.
		      \item Everything that is beautiful is beautiful.
		      \item Every philosopher is happy. So if everything is a philosopher, everything is happy.
		      \item Everything is a philosopher and everything is happy. It follows that everything is a happy philosopher.
		      \item Something is such that it is happy if Ella is a philosopher. So if Ella is a philosopher, then something is happy.
		      \item There is a beautiful country. And so something is beautiful and something is a country.
		      \item Nothing is ugly, and so everything is not ugly.
		      \item Something is not right. It follows that not everything is right.
		      \item Not everything is free. And so something is not free.
		      \item Everything is not free. It follows that nothing is free.
		      \item Every philosopher is wise, and everything wise is happy. Thus, every philosopher is happy.
		      \item Every philosopher is happy. There is a wise philosopher. And something is wise and happy.
		      \item Everything loves everything. Thus, everything loves itself.
		      \item Something loves itself. And so something loves something.
		      \item Nothing loves something which returns its loves.
		            %\item Some of the people can be all right part of the time, but all the people can't be all right all the time.
	      \end{enumerate}
	\item[\bf Deduction:] Use the natural deduction rules $\mathcal{R}_1^+$ to prove that the conclusion of each of the regimented arguments above follows from its premises.
	      %\begin{enumerate}[label=(\arabic*)]
	      %\item $\forall xPx\vdash \forall yPy$.
	      %\item $\vdash \forall x(Px\rightarrow Px)$.
	      %\item $\forall x(Pa\rightarrow Qx)\vdash Pa\rightarrow\forall xQx$.
	      %\item $\forall x(Px\rightarrow Qx)\vdash \forall Px\rightarrow\forall xQx$.
	      %\item $\forall xPx\wedge\forall xQx\vdash \forall y(Py\wedge Qy)$.
	      %\item $\exists x(Pa\rightarrow Qx)\vdash Pa\rightarrow\exists xQx$.
	      %\item $\exists x(Px\wedge Qx)\vdash \exists yPy\wedge \exists yQy$.
	      %\item $\neg\exists xPx\vdash \forall y\neg Py$.
	      %\item $\exists x\neg Px\vdash \neg\forall yPy$.
	      %\item $\neg\forall xPx\vdash\exists y\neg Py$.
	      %\item $\forall x\neg Px\vdash \neg\exists yPy$.
	      %\item $\forall x(Qx\rightarrow Rx), \forall x(Px\rightarrow Qx)\vdash\forall x(Px\rightarrow Rx)$.
	      %\item $\forall x(Qx\rightarrow Rx),\exists x(Px\wedge Qx)\vdash\exists x(Px\wedge Rx)$.
	      %\item $\forall x\forall yRxy\vdash\forall xRxx$.
	      %\item $\exists xRxx\vdash\exists x\exists yRxy$.
	      %\item $\neg\exists x\exists y(Rxy\wedge Ryx)\vdash\neg\exists xRxx$.
	      %\end{enumerate}
	\item[\bf Metalogic:] Prove that every theorem of $\mathcal{F}^+$ is also a theorem of $\mathcal{F}^+_1$.
	\item[\bf Bonus:] Regiment the following into $\FI$:
	      \begin{enumerate}[label=(\arabic*)]\small
		      \item Everybody loves somebody.
		      \item Everybody everybody loves loves somebody.
		      \item Everybody everybody everybody loves loves loves somebody.
		            %\item Everybody everybody everybody everybody loves loves loves loves somebody.
		      \item You can fool all the people some of the time, and some of the people all the time, but you cannot fool all the people all the time.
	      \end{enumerate}
\end{enumerate}







%\section*{\sc Quantified Modal Logic: Syntax}

%\section*{\sc Quantified Modal Logic: Axiomatic Systems}

%\section*{\sc Quantified Modal Logic: Semantics}

